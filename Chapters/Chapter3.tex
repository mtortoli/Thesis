\chapter{Capitolo 3: Albero Evolutivo}
Una delle sfide più importanti della bioinformatica, nonché l'obiettivo principale della filogenetica, è la costruzione degli alberi evolutivi.
\newline
Ricordando la definizione di albero, ovvero un grafo non orientato connesso e aciclico \cite{algoritmiEStruttureDati2}, \textit{L'albero evolutivo} o \textit{albero filogenetico} è un diagramma che rappresenta le relazioni evolutive tra i vari organismi \cite{buildingaphylogenictree}. La sua peculiarità consiste nel poterli costruire in base a dati genetici, genomici o morfologici, affinché si possano descrivere le relazioni che vi sono tra organismi viventi oppure tra specie estinte e specie viventi.
\newline













\newpage
Ordine dei capitoli:
\begin{itemize}
	\item Cap 3: alberi evolutivi;
	\item Cap 4: albero additivo;
	\item Cap 5: UPGMA (Unweighted Pair Group Method with Arithmetic Mean)
	\item cap 6: NEIGHBORJOINING
\end{itemize}

