\chapter{Capitolo 1: Concetti base di biologia}
La bioinformatica è una materia che tratta tanto l'informatica quanto la biologia, pertanto è necessario illustrarne gli argomenti più importanti, che verranno spiegati in modo funzionale allo scopo della presente tesi.
\newline
La \textit{biologia} è la scienza che studia la vita, dagli attori che ne fanno parte fino ai processi in cui essi sono coinvolti. Poiché la vita nella terra si estende dalla biosfera fino alle molecole, si è resa necessaria l'esigenza di trovare una vera e propria scala a livello globale. Tra le varie forme di vita si trovano, omettendo quelle non interessate ed in ordine crescente, le molecole (insiemi di atomi), le macromolecole (insieme di molecole) e le cellule (insieme di macromolecole).
\newline
Ci sono quattro tipi di macromolecole che risultano essenziali per tutte le forme di vita:
\begin{itemize}
	\item \textit{Polisaccaridi}: macromolecole formate da un'insieme molecole, ovvero i monosaccaridi, tra cui il fruttosio, il glucosio e così via.
	\item \textit{Proteine}: sono la "struttura" degli esseri viventi, infatti consentono lo sviluppo e mantenimento degli organi.
	\item \textit{Lipidi}: chiamati anche grassi, sono le riserve di energia.
	\item \textit{Acidi nucleici}: DNA e RNA.
\end{itemize}
Di seguito vengono approfonditi il DNA e l'RNA.

\section{DNA}