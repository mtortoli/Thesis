\chapter{Capitolo 1: La bioinformatica}
Per molti anni l'informatica è stata una scienza a sé stante, tuttavia negli ultimi decenni, grazie al progresso scientifico e tecnologico, sono nate nuove discipline chiamate genericamente \textbf{X-Informatics}. Queste sono il risultato dell'incontro tra l'informatica ed altre scienze di base (quali la biologia, la chimica, l'astronomia, la geologia etc) e tra queste citiamo la bioinformatica, la chemioinformatica, l'astroinformatica, la geoinformatica e così via.
Anche se queste discipline sono diverse tra loro, ad esempio i dati raccolti in campo astronomico saranno di natura diversa rispetto quelli raccolti in campo biologico, condividono gli stessi obiettivi, come riportato nella pubblicazione \cite{xinformatics} \textit{X-Informatics: Practical Semantic Science}:
\begin{itemize}
	\item Processamento ed estrazione delle informazioni
	\item Utilizzo trasparente ed efficiente dei dati in base al contesto scientifico, dalla raccolta, all'analisi fino alla catagolazione
	\item Integrazione di dati ottenuti tra sorgenti eterogenee
	\item Interazione con la raccolta dati adattata e personalizzata all'utente
	\item Fornire supporto decisionale per l'utente, riducendo così i possibili errori e facilitando l'analisi dei risultati
\end{itemize}
Tra tutte queste discipline, risulta di particolare importanza la bioinformatica.
\section{Che cosa \`e la bioinformatica?}
Non esiste un unico modo con il quale definire la bioinformatica, infatti è possibile trovare definizioni diverse tra loro in quanto i professionisti non sempre concordano sulla portata del suo uso sia nel campo della biologia che dell'informatica. Tuttavia una possibile definizione è la seguente:
\begin{center}
\textit{La bioinformatica è un campo multidisciplinare della scienza che coinvolge la genetica, la biologia molecolare, l'informatica, la matematica e la statistica, il cui obiettivo è quello di studiare sistemi biologici a livello molecolare e cellulare utilizzando metodi e modelli informatici e computazionali.}
\end{center}
A prescindere dalla natura del problema da affrontare, è possibile individuare un approccio standard, suddiviso in cinque steps:
\begin{itemize}
	\item Studio ed analisi del problema da affrontare
	\item Collezionamento ed analisi di dati statistici a fronte di dati bioligici in input 
	\item Creazione di modelli ed uso di strumenti matematici che possano essere applicati al problema in esame, al fine di sviluppare un algoritmo
	\item Creazione, valutazione e test dell'algoritmo risolutivo del problema
\end{itemize}

\subsection{Subsection 1.1.1}
A subsection
\subsubsection{Subsubsection 1.1.1.1}
A subsubsection with a list of elements
\begin{itemize}
	\item element1
	\item element2
	\item element3
\end{itemize}
\paragraph{paragraph} A paragraph