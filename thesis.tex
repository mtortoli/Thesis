%--------------------------------------------------------------
% tesi.tex 
%--------------------------------------------------------------
% Corso di Laurea in Informatica 
% http://if.dsi.unifi.it/
% @Facolt\`a di Scienze Matematiche, Fisiche e Naturali
% @Universit\`a degli Studi di Firenze
%--------------------------------------------------------------
% - template for the main file of Informatica@Unifi Thesis 
% - based on Classic Thesis Style Copyright (C) 2008 
%   Andr\'e Miede http://www.miede.de   
%--------------------------------------------------------------

\documentclass[twoside,openright,titlepage,fleqn,
,	headinclude,12pt,a4paper,BCOR5mm,footinclude,table]{scrbook}
%--------------------------------------------------------------
\newcommand{\myItalianTitle}{Applicazioni dell'algoritmica alla biologia: evolutionary trees e clustering\xspace}
\newcommand{\myEnglishTitle}{Applications of algorithmics to biology: evolutionary trees and clustering\xspace}
\newcommand{\myDegree}{Corso di Laurea in Informatica\xspace}
\newcommand{\myName}{Matteo Tortoli\xspace}
\newcommand{\myProf}{Maria Cecilia Verri\xspace}
\newcommand{\myOtherProf}{Nome Cognome\xspace}
\newcommand{\mySupervisor}{Nome Cognome\xspace}
\newcommand{\myFaculty}{
	Scuola di Scienze Matematiche, Fisiche e Naturali\xspace}
\newcommand{\myUni}{\protect{
	Universit\`a degli Studi di Firenze}\xspace}
\newcommand{\myLocation}{Firenze\xspace}
\newcommand{\myTime}{Anno Accademico 2018-2019\xspace}
\newcommand{\myVersion}{Version 0.1\xspace}
%--------------------------------------------------------------

\usepackage[italian, english]{babel}
\usepackage[utf8]{inputenc} 
\usepackage[T1]{fontenc} 
\usepackage[square,numbers]{natbib} 
\usepackage[fleqn]{amsmath}  
\usepackage{ellipsis}
\usepackage{listings}
\usepackage{subfig}
\usepackage{caption}
\usepackage{appendix}
\usepackage{siunitx}
\usepackage[hyphens]{url}

%--------------------------------------------------------------
\usepackage{dia-classicthesis-ldpkg}
%--------------------------------------------------------------


%
% Options for classicthesis.sty:
% tocaligned eulerchapternumbers drafting linedheaders 
% listsseparated subfig nochapters beramono eulermath parts 
% minionpro pdfspacing
\usepackage[eulerchapternumbers,linedheaders,subfig,beramono,eulermath,
parts]{classicthesis}
%--------------------------------------------------------------
\newlength{\abcd} % for ab..z string length calculation
% how all the floats will be aligned
\newcommand{\myfloatalign}{\centering} 
\setlength{\extrarowheight}{3pt} % increase table row height
\captionsetup{format=hang,font=small}
%--------------------------------------------------------------
% Layout setting
%--------------------------------------------------------------
\usepackage{geometry}
\geometry{
	a4paper,
	ignoremp,
	bindingoffset = 1cm, 
	textwidth     = 13.5cm,
	textheight    = 21.5cm,
	lmargin       = 3.5cm, % left margin
	tmargin       = 4cm    % top margin 
}




%%
%% Julia definition (c) 2014 Jubobs
%%
\lstdefinelanguage{Julia}%
  {morekeywords={abstract,break,case,catch,const,continue,do,else,elseif,%
      end,export,false,for,function,immutable,import,importall,if,in,%
      macro,module,otherwise,quote,return,switch,true,try,type,typealias,%
      using,while},%
   sensitive=true,%
   alsoother={},%
   morecomment=[l]\#,%
   morecomment=[n]{\#=}{=\#},%
   morestring=[s]{"}{"},%
   morestring=[m]{'}{'},%
}[keywords,comments,strings]%

\lstset{%
    language         = Julia,
    basicstyle       = \ttfamily,
    keywordstyle     = \bfseries\color{blue},
    stringstyle      = \color{magenta},
    commentstyle     = \color{ForestGreen},
    showstringspaces = false,
}
%%%

\usepackage{tikz}
\usetikzlibrary{arrows}
\usetikzlibrary{positioning}
\tikzset{main node/.style={circle,fill=blue!20,draw,minimum size=1cm,inner sep=0pt},
            }



%--------------------------------------------------------------
\begin{document}
\frenchspacing
\raggedbottom
\pagenumbering{roman}
\pagestyle{plain}
%--------------------------------------------------------------
% Frontmatter
%--------------------------------------------------------------


%--------------------------------------------------------------
% titlepage.tex (use thesis.tex as main file)
%--------------------------------------------------------------
\begin{titlepage}
	\begin{center}
   	\large
      \hfill
      \vfill
      \begingroup
         \includegraphics[scale=0.15]{logo/LOGO}\\
%			\spacedallcaps{\myUni} \\ 
			\myFaculty \\
			\myDegree \\ 
			\vspace{0.5cm}
         \vspace{0.5cm}    
           
      \endgroup 
      \vfill 
      \begingroup
      	\color{Maroon}\spacedallcaps{\myItalianTitle} \\ $\ $\\
      	\spacedallcaps{\myEnglishTitle} \\ 	
	\bigskip
      \endgroup
      \spacedlowsmallcaps{\myName} \\ $\ $\\
      \spacedlowsmallcaps{\myProf}
      \vfill 
      \vfill
    
      \vfill
      \vfill
      \myTime
      \vfill                      
	\end{center}        
\end{titlepage}   
%--------------------------------------------------------------
% back titlepage
%--------------------------------------------------------------
   \newpage
	\thispagestyle{empty}
	\hfill
	\vfill
	\noindent\myName: 
	\textit{\myItalianTitle,} 
	\myDegree, \textcopyright\ \myTime
%--------------------------------------------------------------
% back titlepage end
%--------------------------------------------------------------

\pagestyle{scrheadings}
%--------------------------------------------------------------
% Mainmatter
%--------------------------------------------------------------
\pagenumbering{arabic}
% use \cleardoublepage here to avoid problems with pdfbookmark
%\include{intro} % use \myChapter command instead of \chapter
%\cleardoublepage\myPart{Part I}
%\include{chapter01}
%\cleardoublepage\myPart{Part II}
%\include{chapter02}
%\include{chapter03}
\tableofcontents
\listoftables
\listoffigures
\chapter{Chapter 1}
Empty thesis model
\section{Section 1.1}
A section
\subsection{Subsection 1.1.1}
A subsection
\subsubsection{Subsubsection 1.1.1.1}
A subsubsection with a list of elements
\begin{itemize}
	\item element1
	\item element2
	\item element3
\end{itemize}
\paragraph{paragraph} A paragraph


\begin{thebibliography}{50}

\bibitem{laBiologiaStrutturaleInMovimento}
Allegretti M. (2014).\newline
\textit{La biologia strutturale in movimento}. \newline
AIRInforma: AIRIcerca.
\\\texttt{\url{http://informa.airicerca.org/it/2014/10/04/biologia-strutturale-in-movimento/}}

\bibitem{SystemsBiologyTheNextFrontierforBioinformatics}
Bacic, A., Likić, V. A., Lithgow, T., McConville, M. J.(2010).\newline 
\textit{Systems Biology: The Next Frontier for Bioinformatics}.\newline
Advances in Bioinformatics, 2010, 1–10. \newline
DOI: \\\texttt{\url{l10.1155/2010/268925}}

\bibitem{clinicalApplicationOfBioinformatics}
Bayat Ardeshir (2002).\newline
\textit{Science, medicine, and the future: Bioinformatics}.\newline
BMJ, 324(7344), 1018–1022.
DOI: \\\texttt{\url{10.1136/bmj.324.7344.1018}}

\bibitem{xinformatics}
Borne, K.~D.\newline
\textit{X-Informatics: Practical Semantic Science}.
\\\texttt{\url{http://adsabs.harvard.edu/abs/2009AGUFMIN43E..01B}}. \newline
George Mason University Fairfax VA, American Geophysical Union Fall Meeting, 12/2009.

\bibitem{WhatissystemsbiologyFrontiersinPhysiology}
Breitling, R. (2010). \newline
\textit{What is systems biology? Frontiers in Physiology}.\newline
DOI: \\\texttt{\url{10.3389/fphys.2010.00009}}

\bibitem{computationalApproachForProteinStructurePrediction}
Candavelou, M., Gollapalli, S., Gopal,J., Karthikeyan, K., Venkatesan, A.(2013).\newline \textit{Computational Approach for Protein Structure Prediction}.\newline
Healthcare Informatics Research, 19(2), 137. \newline
DOI: \\\texttt{\url{10.4258/hir.2013.19.2.137}}

\bibitem{bioinfomaticsdefcan}
Can T. (2013).\newline
\textit{Introduction to Bioinformatics}.\newline
miRNomics: MicroRNA Biology and Computational Analysis.\newline
DOI: \\\texttt{\url{10.1007/978-1-62703-748-8_4}}

\bibitem{phylogeneticsAnIntroduction}
Emery L.\newline
\textit{Phylogenetics: An introduction}.\newline
European Bioinformatics Institute-European Molecular Biology Laboratory.
\\\texttt{\url{https://www.ebi.ac.uk/training/online/course/introduction-phylogenetics}}\newline
Oxford.

\bibitem{phylogenetics}
Haque Sultan Omar (2016).\newline
\textit{Phylogenetics}.\newline
Enciclopedia Britannica.
\\\texttt{\url{https://www.britannica.com/science/phylogenetics\#accordion-article-history}}

\bibitem{icarcnribioinfdefinition} 
ICAR CNR: Istituto Di Calcolo E Reti Ad Alte Prestazioni.\newline
\textit{Bioinformatica}.
\\\texttt{\url{https://www.icar.cnr.it/bio-informatica/}}

\bibitem{sequenceClusteringInBioinformaticsAnEmpiricalStudy}
Jiang, X., Lin, G., Liu, X., Zeng, X., Zou, Q.(2018).\newline
\textit{Sequence clustering in bioinformatics: an empirical study}.\newline
Briefings in Bioinformatics.\newline
DOI: \\\texttt{\url{https://doi.org/10.1093/bib/bby090}}

\bibitem{introductionbioinfalg} 
Jones C. Neil and Pevzner A. Pavel.\newline
\textit{An introduction to bioinformatics algorithms}.\newline
Massachusetts, Massachusetts Institute of Technology, 2004.

\bibitem{mathworksWhatIsTheGeneticAlgorithm}
MathWorks.\newline
\textit{What Is the Genetic Algorithm?}
\\\texttt{\url{https://it.mathworks.com/help/gads/what-is-the-genetic-algorithm.html}}

\bibitem{introductionToGeneticAlgorithmsIncludingExampleCode}
Mallawaarachchi V (2017).\newline
\textit{Introduction To Genetic Algorithms-Including Example Code}.
\\\texttt{\url{https://towardsdatascience.com/introduction-to-genetic-algorithms-including-example-code-e396e98d8bf3}}

\bibitem{bioinformaticsSequenceAndGenomeAnalysis}
Mount W. David.\newline
\textit{Bioinformatics: Sequence and Genome Analysis}.\newline
Cold Spring Harbor Laboratory Press, 2004.

\bibitem{dnaSequencingFactSheet}
National Human Genome Research Institute (2015).\newline
\textit{DNA Sequencing Fact Sheet}.
\\\texttt{\url{https://www.genome.gov/about-genomics/fact-sheets/DNA-Sequencing-Fact-Sheet}}

\bibitem{proteineDataBank}
Proteine Data Bank Website.\newline
\textit{PDB}.
\\\texttt{\url{http://www.rcsb.org/}}

\bibitem{bioinformaticsTrendsInGeneExpressionAnalysis}
Raut, A., Raut, S. A., Sathe, S. R.(2010). \newline
\textit{Bioinformatics: Trends in gene expression analysis}.\newline
International Conference on Bioinformatics and Biomedical Technology.\newline
DOI:\\\texttt{\url{10.1109/icbbt.2010.5479003}}

\bibitem{SBW}
Systems Biology Workbench Website.\newline
\textit{SBW}.
\\\texttt{\url{http://sbw.sourceforge.net/}}

\bibitem{phylogeneticsOxford}
Semple C., Steel M.\newline
\textit{Phylogenetics}.\newline
Oxford, Oxford University Press, 2003.

\bibitem{SBML}
Systems Biology Markup Language.\newline
\textit{SBML}.
\\\texttt{\url{http://sbml.org/}}

\bibitem{populationGeneticsAndMicroevolutionaryTheory}
Templeton R. Alan.\newline
\textit{Population Genetics and Microevolutionary Theory}.\newline
Washington University, St. Louis, Missouri, Wiley-Liss, 2006.

\bibitem{treccanibioinf} 
Tramontano Anna (2003).\newline
\textit{La grande scienza. Bioinformatica}.
\\\texttt{\url{http://www.treccani.it/enciclopedia/la-grande-scienza-bioinformatica_\%28Storia-della-Scienza\%29/}}

\end{thebibliography}


%--------------------------------------------------------------
\end{document}
%--------------------------------------------------------------