\chapter{Capitolo 4: Algoritmi basati sulla distanza}
In questo capitolo vengono illustrati i principali algoritmi utilizzati per la costruzione degli alberi evolutivi senza radice. L'obiettivo è quello di trovare una soluzione al cosiddetto \textit{problema degli alberi basati sulla distanza}, ma prima di ciò è necessario introdurre alcuni concetti, tra cui la matrice delle distanze.

\section{Matrice delle distanze}
Dati due punti, $x$ e $y$, la \textit{distanza} può essere vista come loro "lontananza" in uno spazio $k$-dimensionale. Nella fattispecie, la distanza è una funzione $d(x,y)$ che possiede le seguenti proprietà \cite{molaCagliari}:
\begin{enumerate}
	\item \textit{non negatività}:
	\[d(x,y)\geq 0\hspace{2em} \forall \: x,y\in R^k\]
	\item \textit{identità}:
	\[d(x,y)=0 \; \leftrightarrow \; x=y\]
	\item \textit{simmetria}:
	\[d(x,y)=d(y,x)\hspace{2em} \forall \: x,y\in R^k\]
	\item \textit{disuguaglianza triangolare}:
	\[d(x,y)\leq d(x,z)+d(y,z)\hspace{2em} \forall \: x,y,z\in R^k\]
\end{enumerate}
TODO: parlare del fatto che ci sono diversi modi per calcolare la distanza (inserire quella euclidea se interessata), poi introdurre matrice delle distanze, le proprietà ed un esempio!
\newpage

\begin{center}
\textbf{Problema degli alberi basati sulla distanza:}
\newline
\textit{Dato in \textbf{input} una matrice delle distanze si ottiene in \textbf{output} un albero evolutivo adattato alla matrice stessa.}
\end{center}
Denotando con $d_i,_j(T)$ la distanza evolutiva tra la 
Si dice che un albero $T$ si \textit{adatta} ad una matrice delle distanze $D$ se per ogni coppia di foglie $i$ e $j$ si ha che $d_i,_j(T)=D_i,_j$.