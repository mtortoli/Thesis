\chapter{Capitolo 1: La bioinformatica}
Per molti anni l'informatica è stata una scienza a sé stante, tuttavia negli ultimi decenni, grazie al progresso scientifico e tecnologico, sono nate nuove discipline chiamate genericamente \textbf{X-Informatics}. Queste sono il risultato dell'incontro tra l'informatica ed altre scienze di base (quali la biologia, la chimica, l'astronomia, la geologia etc) e tra queste citiamo la bioinformatica, la chemioinformatica, l'astroinformatica, la geoinformatica e così via.
Anche se queste discipline sono diverse tra loro, ad esempio i dati raccolti in campo astronomico saranno di natura diversa rispetto quelli raccolti in campo biologico, condividono gli stessi obiettivi, come riportato nella pubblicazione \cite{xinformatics} \textit{X-Informatics: Practical Semantic Science}:
\begin{itemize}
	\item Processamento ed estrazione delle informazioni
	\item Utilizzo trasparente ed efficiente dei dati in base al contesto scientifico, dalla raccolta, all'analisi fino alla catagolazione
	\item Integrazione di dati ottenuti tra sorgenti eterogenee
	\item Interazione con la raccolta dati adattata e personalizzata per l'utente
	\item Fornire supporto decisionale per l'utente, riducendo così i possibili errori e facilitando l'analisi dei risultati
\end{itemize}
Tra tutte queste discipline, risulta di particolare importanza la bioinformatica.
\section{Che cosa \`e la bioinformatica?}
Non esiste un unico modo con il quale definire la bioinformatica, infatti è possibile trovare definizioni diverse tra loro in quanto i professionisti non sempre concordano sulla portata del suo uso sia nel campo della biologia che dell'informatica. Tuttavia una possibile definizione è la seguente:
\begin{center}
\textit{La bioinformatica è un campo multidisciplinare della scienza che coinvolge la genetica, la biologia molecolare, l'informatica, la matematica e la statistica, rivolta a studiare sistemi biologici utilizzando metodi e modelli informatici e computazionali.}
\end{center}
Tra i vari obiettivi precedentemente elencati nell'introduzione al capitolo, va aggiunto quello che risulta l'obiettivo principale di questa disciplina, ovvero quello di aumentare la conoscenza di tutti quei processi di natura biologica.
\newline
A prescindere dalla natura del problema da affrontare, è possibile individuare un approccio standard, suddiviso in cinque steps:
\begin{itemize}
	\item Studio ed analisi del problema da affrontare
	\item Collezionamento ed analisi di dati statistici a fronte di dati biologici in input 
	\item Creazione di modelli ed uso di strumenti matematici che possano essere applicati al problema in esame, al fine di sviluppare un algoritmo
	\item Creazione, valutazione e test dell'algoritmo risolutivo del problema
\end{itemize}
Una parte fondamentale della bioinformatica consiste in esperimenti che generano dati ad alto throughput (high-throughput data), tra cui la misurazione dei modelli di espressione genica oppure la determinazione della sequenza genomica. Per \textbf{high-throughput data} si intendono quei dati biologici ottenuti tramite tecniche automatizzate e quindi non ottenibili attraverso metodi convenzionali.
Il mining di questi dati può portare a nuove scoperte scientifiche non solo in campo biologico, ma anche medico, sia nel breve che nel lungo periodo.
\newline
Nel breve periodo, ad esempio grazie al \textit{progetto genoma umano}\footnote{Il progetto genoma umano (Human Genome Project) è stato uno dei più grandi progetti scientifici degli ultimi anni. L'obiettivo era quello di ottenere la sequenza del genoma umano (e quindi il suo intero DNA) e identificare in esso i geni contenuti. Il progetto è cominciato nel 1990, per poi essere completato nel 2003 ed ulteriori ricerche sono ancora in corso.}, si possono scoprire nuovi geni legati alle malattie e nuovi bersagli molecolari, ovvero quei processi biologici, intesi come proteine, recettori, pathway biochimici etc su cui si può intervenire per modificare il decorso di una malattia.
\newline
Nel lungo periodo sarà possibile scoprire eventuali reazioni avverse ai farmaci da individuo ad individuo in base a dei tests, al punto tale che, grazie all'informazione genetica ottenuta attraverso strumenti strumenti informatici, sarà possibile personalizzare l'uso di farmaci, portando ad una migliore efficacia alla terapia individuale, riducendo o addirittura eliminando possibili effetti collaterali.

\section{Aree di ricerca}
Data la natura eterogenea dei dati biologici, la bioinformatica comprende un vasto numero di aree di ricerca in continua crescita, di seguito presentate.

\subsection{Analisi dei genomi}
Uno dei principali focus della bioinformatica riguarda l'analisi dei genomi\footnote{Per genoma si intende l'intero materiale genetico di un organismo, composto da DNA o RNA.} degli organismi il cui sequenziamento\footnote{Il sequenziamento è un processo mediante il quale viene determinata la struttura primaria delle macromolecole (composizione atomica e legami), quali il DNA, RNA e proteine.} è già stato completato, dal moscerino della frutta fino all'essere umano. L'analisi dei genomi è un'area di ricerca relativa non solo alla bioinformatica, ma anche alla genomica, ovvero quella disciplina che studia la struttura, il contenuto, la funzione e l'evoluzione del genoma.
Ma perché analizzare i genomi? un gene viene sequenziato per conoscere la sua funzione ed eventualmente per modificarne la sua funzione. La conoscenza dell'intero genoma di un organismo fornisce le sequenze di tutti i suoi geni, permettendo così di identificare e manipolare i geni importanti che influenzano il metabolismo, la differenziazione e lo sviluppo cellulare e i processi patologici negli umani, animali e nelle piante.
\newline
L'obiettivo in questa area di ricerca per la bioinformatica è quello di identificare e alterare tutti quei geni che abbiano una particolare funzione biologica attraverso strumenti computazionali.

\subsection{Analisi di sequenze}
L'analisi delle sequenze di DNA, RNA o proteine è un processo mediante il quale tali macromolecole vengono sottoposte a dei metodi analitici al fine di capirne la struttura e le funzionalità.
\newline
Di particolare importanza risulta lo studio delle sequenze di DNA\footnote{Il sequenziamento del DNA consiste nel determinare la sequenza di nucleotidi all'interno di un suo frammento.} che possono essere memorizzate in un computer attraverso una vasta varietà di metodi.
La memorizzazione avviene attraverso l'uso di caratteristiche identificative di una determinata sequenza di DNA, ad esempio il nome di un gene oppure la fonte, dopodiché vengono salvate all'interno di database che prendono il nome di \textit{database biologici} (vedere sottosezione 1.2.9).
\newline
L'analisi delle sequenze di DNA permette di venire a conoscenza dell'informazione genetica che viene trasportata all'interno di un suo segmento, grazie al quale, ad esempio, è possibile individuare i cambiamenti di un gene che possono causare una potenziale malattia.
\newline
Una volta estratto il DNA, vengono create migliaia e migliaia di copie di un singolo frammento, in quanto non è singolo frammento non risulterebbe sufficiente. Questi campioni vengono inseriti in dei macchinari chiamati \textit{DNA Sequencer} che svolgono il compito di sequenziamento (del DNA) in modo automatico, infine i dati vengono raccolti ed analizzati.
\newline
Tra i vari algoritmi utilizzati per l'analisi delle sequenze risultano di particolare importanza gli algoritmi di clustering, il cui obiettivo è quello di raggruppare i dati delle sequenze in modo veloce e preciso. 

\subsection{Filogenetica}
\subsection{Bioinformatica strutturale}
\subsection{Espressione genica}
\subsection{Genetica delle popolazioni}
\subsection{Biologia dei sistemi}
\subsection{Data mining}
\subsection{Database biologici} %http://www.cusmibio.unimi.it/scaricare/linkdocenti_09.pdf
\subsection{Bioimmagini}
%https://academic.oup.com/bioinformatics



\section{Storia}
